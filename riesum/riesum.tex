\documentclass[12pt,a4paper]{article}
\usepackage{lipsum}
\usepackage{authblk}
\usepackage[top=2cm, bottom=2cm, left=2cm, right=2cm]{geometry}
\usepackage{fancyhdr}
\usepackage{amsmath}
\usepackage{amssymb}

% -- Begin code formatting --
\usepackage{listings}
\usepackage{courier}
\lstset{
language=C,
basicstyle=\small\sffamily,
numbers=left,
numberstyle=\tiny,
frame=tb,
columns=fullflexible,
showstringspaces=false,
literate={PI}{{$\pi \approx$}}1
}
% -- End code formatting --

\newcommand{\dd}{\,\textrm{d}}

\pagestyle{fancy}

\renewenvironment{abstract}{
\hfill\begin{minipage}{0.95\textwidth}
\rule{\textwidth}{1pt}}
{\par\noindent\rule{\textwidth}{1pt}\end{minipage}}

\makeatletter
\renewcommand\@maketitle{
\hfill
\begin{minipage}{0.95\textwidth}
\vskip 2em
\let\footnote\thanks 
{\LARGE \@title \par }
\vskip 1.5em
{\large \@author \par}
\end{minipage}
\vskip 1em \par
}
\makeatother

\setlength{\parskip}{\baselineskip}
\setlength{\parindent}{0pt}

\begin{document}


\title{Using Riemann Sums to Approximate $\pi$}
\author[1]{Joshua Williams}
\affil[1]{jodawill@members.fsf.org}

\maketitle

\begin{abstract}
Demonstration of a slow, but creative new method of approximating $\pi$ by computing the Riemann sum over a semicircle.
\end{abstract}
\section{Introduction}
Let $n \in \mathbb{Z}$. Then $\pi \approx \frac{8}{n^2} \sum\limits_{i=1}^n \sqrt{i(n-i)}$.

This method was originally described in a paper I wrote on November 30, 2012. I discovered this method during my first semester of calculus at IUK.

\section {Proof}
A semicircle can be represented by the function $f(x) = \sqrt{r^2-x^2}$. The area of a semicircle is $A = \frac{\pi}{2} r^2$, which can also be represented as an integral:
\begin{align*}
\frac{\pi}{2}r^2 &= \int\limits_{-r}^r \sqrt{r^2-x^2}
\end{align*}
Now we write this definite integral in terms of a Riemann sum, where $n \in \mathbb{Z}^+$.
\begin{align*}
\int\limits_{-r}^r \sqrt{r^2-x^2} &= \lim\limits_{n \to \infty} \sum\limits_{i=1}^n  f(x_i^*) (x_i - x_{i-1})\\
&= \lim\limits_{n \to \infty} \sum\limits_{i=1}^n \frac{2r}{n} \sqrt{r^2 - \left(\frac{2i}{n} -r\right)^2}
\end{align*}
It then follows that:
\begin{align*}
\frac{\pi}{2}r^2 &= \lim\limits_{n \to \infty} \sum\limits_{i=1}^n \frac{2r}{n} \sqrt{r^2 - \left(\frac{2i}{n} -r\right)^2}
\end{align*}
Let $r = 1$. Since $n > 0$,
\begin{align*}
\frac{\pi}{2} &= \lim\limits_{n \to \infty} \sum\limits_{i=1}^n \frac{2}{n} \sqrt{1 - \left(\frac{2i}{n} -1\right)^2}\\
\implies \pi &= \frac{4}{n} \lim\limits_{n \to \infty} \sum\limits_{i=1}^n \sqrt{1 - \left(\frac{2i}{n} -1\right)^2}\\
&= \frac{4}{n} \lim\limits_{n \to \infty} \sum\limits_{i=1}^n \sqrt{1-\left( \frac{4i^2}{n^2} - \frac{4i}{n} + 1 \right)}\\
&= \frac{4}{n} \lim\limits_{n \to \infty} \sum\limits_{i=1}^n \sqrt{\frac{4i}{n} - \frac{4i^2}{n^2}}\\
&= \frac{4}{n} \lim\limits_{n \to \infty} \sum\limits_{i=1}^n \sqrt{\frac{4in - 4i^2}{n^2}}\\
&= \frac{4}{n} \lim\limits_{n \to \infty} \sum\limits_{i=1}^n \frac{2}{n} \sqrt{i(n-i)}\\
&= \frac{8}{n^2} \lim\limits_{n \to \infty} \sum\limits_{i=1}^n \sqrt{i(n-i)}
\end{align*}

So to approximate the value of $\pi$, we take a rectangular width of $\frac{1}{n}$ and compute the sum $\frac{8}{n^2} \sum\limits_{i=1}^n \sqrt{i(n-i)}$, as illustrated by the C code in the following section.

\newpage
\section{Example}
\begin{lstlisting}[title=riesum.c]
#include <stdio.h>
#include <math.h>

int main(int argc, char *argv[]) {
 int n;
 printf("Enter a positive integer. n = ");
 for (scanf("%d",&n); n <= 0; scanf("%d",&n)) {
  while (fgetc(stdin) != '\n');
  printf("Enter a valid integer. n = ");
 }

 double sum = 0;

 for (int i = 1; i <= n; i++) {
  sum += sqrt(i*(n-i))/n;
 }

 // Note that we kept one n in the denominator of the sum to avoid
 // an integer overflow.
 sum *= 8/(double)n;

 printf(PI %f\n",sum);

 return 0;
}
\end{lstlisting}

To compile, run: \fontfamily{qcr}\selectfont{gcc riesum.c -o riesum -lm}

\end{document}
